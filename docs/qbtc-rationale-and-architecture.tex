\documentclass[10pt,twocolumn,a4paper]{article}

% ── Packages ──────────────────────────────────────────────────────────
\usepackage[utf8]{inputenc}
\usepackage[T1]{fontenc}
\usepackage{lmodern}
\usepackage[top=2cm,bottom=2cm,left=1.8cm,right=1.8cm]{geometry}
\usepackage{amsmath,amssymb,amsthm}
\usepackage{graphicx}
\usepackage{booktabs}
\usepackage{enumitem}
\usepackage{hyperref}
\usepackage{xcolor}
\usepackage{titlesec}
\usepackage{fancyhdr}
\usepackage{caption}
\usepackage{float}
\usepackage{parskip}
\usepackage{microtype}
\usepackage{csquotes}
\usepackage{tabularx}
\usepackage{dblfloatfix}          % fixes table* at bottom of page
\usepackage{stfloats}             % improved two-column float placement
\usepackage{balance}              % balance final columns

% ── Compact spacing ──────────────────────────────────────────────────
\setlength{\parskip}{4pt plus 1pt minus 1pt}
\setlength{\columnsep}{0.7cm}
\setlist{nosep,leftmargin=1.2em}
\captionsetup{font=small,labelfont=bf,skip=4pt}

% ── Colours & hyperlinks ─────────────────────────────────────────────
\definecolor{qbtcblue}{RGB}{23,47,77}
\definecolor{qbtcaccent}{RGB}{207,126,43}
\hypersetup{
  colorlinks  = true,
  linkcolor   = qbtcblue,
  citecolor   = qbtcblue,
  urlcolor    = qbtcaccent,
}

% ── Section styling (compact) ────────────────────────────────────────
\titleformat{\section}
  {\large\bfseries\color{qbtcblue}}{\thesection.}{0.5em}{}
\titleformat{\subsection}
  {\normalsize\bfseries\color{qbtcblue!80!black}}{\thesubsection}{0.5em}{}
\titleformat{\subsubsection}
  {\small\bfseries\color{qbtcblue!60!black}}{\thesubsubsection}{0.5em}{}
\titlespacing*{\section}{0pt}{10pt}{4pt}
\titlespacing*{\subsection}{0pt}{8pt}{3pt}
\titlespacing*{\subsubsection}{0pt}{6pt}{2pt}

% ── Header / footer ──────────────────────────────────────────────────
\pagestyle{fancy}
\fancyhf{}
\fancyhead[L]{\footnotesize\textcolor{qbtcblue}{qBTC: A Sovereign Quantum-Safe Blockchain}}
\fancyhead[R]{\footnotesize\textcolor{qbtcblue}{\today}}
\fancyfoot[C]{\footnotesize\thepage}
\renewcommand{\headrulewidth}{0.4pt}

% ── Math environments ────────────────────────────────────────────────
\newtheorem{definition}{Definition}
\newtheorem{property}{Property}

% ══════════════════════════════════════════════════════════════════════
\begin{document}

% ── Title block (inline, no separate page) ───────────────────────────
\twocolumn[{%
\centering
\vspace{0.5cm}
{\LARGE\bfseries\color{qbtcblue} qBTC: A Sovereign Quantum-Safe Blockchain\par}
\vspace{0.3cm}
{\large Rationale, Architecture, and Protocol Specification\par}
\vspace{0.2cm}
{\normalsize Version 1.0 \quad---\quad \today\par}
\vspace{0.15cm}
{\small\textsc{bitcoinqs.org}\par}
\vspace{0.5cm}

\begin{minipage}{0.85\textwidth}
\small
\textbf{Abstract.}
We present \textbf{qBTC}, a sovereign Layer-1 proof-of-work blockchain
that inherits Bitcoin's consensus model---longest chain rule, UTXO
accounting, SHA-256d block headers, difficulty adjustment, and a
21\,M fixed supply---while replacing the ECDSA signature scheme with
\textbf{ML-DSA-87} (CRYSTALS-Dilithium), the lattice-based digital
signature standard published by NIST in FIPS~204.  The motivation is
direct: Shor's algorithm renders every ECDSA-signed UTXO recoverable
by a sufficiently powerful quantum computer, and Bitcoin's governance
process is unlikely to produce a hard-fork migration before the threat
window opens.  Rather than depend on that uncertain timeline, qBTC
constructs the chain that should have existed from the start: one
whose security does not degrade as physics advances.  This paper
describes the threat model, design rationale, protocol architecture,
network topology, and security properties of the system.
\end{minipage}
\vspace{0.6cm}
}]

% ══════════════════════════════════════════════════════════════════════
\section{Introduction}
\label{sec:introduction}
% ══════════════════════════════════════════════════════════════════════

Bitcoin's security rests on two cryptographic pillars: the SHA-256
hash function, which secures proof-of-work and address derivation, and
the Elliptic Curve Digital Signature Algorithm (ECDSA) over
\texttt{secp256k1}, which authorises spending.  Of these, only the
second is vulnerable to quantum attack.

Shor's algorithm~[1] provides a polynomial-time procedure for
computing discrete logarithms on elliptic curves.  A quantum computer
with approximately $2{,}500$ logical qubits---roughly $10^6$--$10^7$
physical qubits under current error-correction overhead---would
recover a \texttt{secp256k1} private key from its public key in hours
rather than the $2^{128}$ classical operations that constitute today's
security margin.

The timeline is contested.  Optimistic estimates place
cryptanalytically relevant quantum computers within the 2030--2035
window; conservative estimates extend to 2040 or beyond.  But the
debate over \emph{when} obscures a more urgent structural problem:
Bitcoin's migration path is extraordinarily narrow.

\subsection{The Migration Problem}

Replacing ECDSA in Bitcoin requires a network-wide hard fork---or, at
minimum, a soft fork introducing a new witness version with
post-quantum verification.  The history of Bitcoin governance suggests
years of deliberation, implementation, and activation.  SegWit took
four years; Taproot took three.

A post-quantum migration would be categorically more complex:

\begin{enumerate}[label=(\roman*)]
  \item \textbf{Signature size.}  ML-DSA-87 signatures are
    $\approx$4{,}627 bytes versus 72 for ECDSA.  Public keys are
    2{,}592 bytes versus 33.  This transforms block-size economics,
    fee markets, and propagation latency.

  \item \textbf{Address migration.}  Every ECDSA-locked UTXO must be
    re-spent to a post-quantum address.  Dormant wallets---including
    Satoshi's estimated 1.1\,M~BTC---remain permanently vulnerable.

  \item \textbf{Consensus risk.}  Hard forks carry the risk of chain
    splits and community fragmentation, as the Bitcoin Cash precedent
    demonstrates.
\end{enumerate}

Even under optimistic assumptions, the window between \emph{plan
announcement} and \emph{migration completion} will span years---years
during which ``harvest now, decrypt later'' becomes increasingly
rational for well-resourced adversaries.

\subsection{The Case for Sovereign Construction}

qBTC rejects the premise that the correct response is to wait for
Bitcoin to upgrade itself.  Instead, it constructs the blockchain that
Bitcoin would be if it were designed today: a sovereign Layer-1 network
with its own genesis block, its own consensus, and its own security
guarantees---built from the ground up on post-quantum cryptography.

The design philosophy is deliberately conservative:

\begin{itemize}
  \item \textbf{Same supply curve.}  21\,M total coins, halving-based
    emission, 8-decimal-place precision.
  \item \textbf{Same consensus model.}  Proof-of-work with SHA-256d,
    longest cumulative-difficulty chain rule, UTXO accounting.
  \item \textbf{Same operational model.}  JSON-RPC mining interface
    compatible with existing pool software.
  \item \textbf{Different cryptography.}  ML-DSA-87 replaces ECDSA at
    every signing and verification point.
\end{itemize}

qBTC is not a fork, not a sidechain, not a Layer-2.  It is a
sovereign chain that inherits Bitcoin's proven economic architecture
while eliminating its single largest cryptographic liability.  Users
do not depend on Bitcoin governance to protect their assets; they hold
keys whose security is not contingent on the pace of quantum hardware.


% ══════════════════════════════════════════════════════════════════════
\section{Cryptographic Foundation}
\label{sec:crypto}
% ══════════════════════════════════════════════════════════════════════

\subsection{ML-DSA-87 Selection Rationale}

NIST's Post-Quantum Cryptography Standardization Process (initiated
2016, primary standards published 2024) produced three signature
schemes: ML-DSA (FIPS~204), SLH-DSA (FIPS~205), and the forthcoming
FN-DSA.  qBTC adopts \textbf{ML-DSA-87}, the highest security level
of the lattice-based scheme:

\begin{enumerate}[label=(\alph*)]
  \item \textbf{Security level.}  NIST Level~5: at least $2^{256}$
    classical and $2^{128}$ quantum security, matching or exceeding
    \texttt{secp256k1}.
  \item \textbf{Performance.}  Sub-millisecond signing and
    verification on commodity hardware.  SLH-DSA imposes 10--100\,ms
    signing, degrading block validation throughput.
  \item \textbf{Maturity.}  FIPS~204 is a final standard with
    reference implementations in \texttt{liboqs}~[9] (C/Python)
    and \texttt{@noble/post-quantum} (JavaScript).
  \item \textbf{Conservative margin.}  Level~5 over Level~2/3
    because the cost of a future security-level upgrade is high,
    while the marginal verification cost is negligible relative to
    SHA-256d proof-of-work.
\end{enumerate}

\subsection{Address Derivation}

\begin{definition}[qBTC Address]
Given an ML-DSA-87 public key $pk$ (1{,}312 bytes):
\begin{align}
  h       &= \mathrm{SHA3\text{-}256}(pk)[0{:}20] \notag\\
  v       &= \mathtt{0x00} \,\|\, h \notag\\
  c       &= \mathrm{SHA3\text{-}256}(v)[0{:}4] \notag\\
  \mathrm{addr} &= \texttt{``bqs1''} \,\|\, \mathrm{Base58}(v \,\|\, c) \notag
\end{align}
\end{definition}

The prefix \texttt{bqs1} (Bitcoin Quantum-Safe, version~1) is
visually distinct from Bitcoin addresses (\texttt{1}, \texttt{3},
\texttt{bc1}).  SHA3-256 is chosen over SHA-256/RIPEMD-160 to avoid
dependence on hash families whose quantum resistance is less
rigorously characterised.

\subsection{Wallet Encryption}

Private keys are encrypted at rest with AES-256-GCM, keyed via
PBKDF2-HMAC-SHA256 (310{,}000 iterations, 16-byte salt, 12-byte IV).
The ML-DSA-87 secret key (4{,}896 bytes) is size-validated at signing
time to prevent truncation or key-confusion attacks.  The wallet file
contains ciphertext, salt, IV, public key, and derived
address---enabling read-only operations without decryption.


% ══════════════════════════════════════════════════════════════════════
\section{Consensus Protocol}
\label{sec:consensus}
% ══════════════════════════════════════════════════════════════════════

\subsection{Proof-of-Work}

qBTC employs SHA-256d proof-of-work over an 80-byte block header
identical in structure to Bitcoin's:

\begin{table}[H]
\centering\small
\caption{Block header (80 bytes)}
\begin{tabular}{@{}lll@{}}
\toprule
\textbf{Field} & \textbf{Size} & \textbf{Encoding} \\
\midrule
Version         & 4\,B  & LE uint32 \\
Prev.\ hash    & 32\,B & LE SHA-256d \\
Merkle root     & 32\,B & LE SHA-256d \\
Timestamp       & 4\,B  & LE uint32 \\
Bits            & 4\,B  & Compact target \\
Nonce           & 4\,B  & LE uint32 \\
\bottomrule
\end{tabular}
\end{table}

Validity requires $\mathrm{SHA256d}(\text{header}) <
\mathrm{target}(\text{bits})$.  Retaining SHA-256d---rather than a
``quantum-resistant'' hash---is deliberate: Grover's algorithm~[5]
provides only a quadratic speedup against preimage search, reducing
$2^{256}$ to $2^{128}$, which remains infeasible.  The vulnerability
is in the signature scheme, not the hash function.

\subsection{Difficulty Adjustment}

Target block time: \textbf{10~seconds}.  Every $120{,}960$ blocks
($\approx$14~days), the target is recalculated:
\[
  T_{\text{new}} = T_{\text{cur}} \cdot
    \frac{t_{\text{actual}}}{t_{\text{expected}}}
  \;,\quad
  \tfrac{1}{4} \leq \frac{T_{\text{new}}}{T_{\text{cur}}} \leq 4
\]
The clamping bound prevents oscillatory instability.  The interval
preserves Bitcoin's two-week adjustment rhythm.

\subsection{Chain Selection and Reorgs}

Fork resolution uses cumulative difficulty (total work), not
longest chain by height.  Reorganisations are capped at
\textbf{100~blocks} to prevent deep-reorg attacks.  Reorgs execute
atomically via RocksDB \texttt{WriteBatch} with a persistent
completion marker enabling deterministic crash recovery.

\subsection{Merkle Tree}

Transactions are committed via a standard binary Merkle tree using
SHA-256d, with final-hash duplication for odd counts (consistent with
Bitcoin).  The root is embedded in the block header and verified during
validation.


% ══════════════════════════════════════════════════════════════════════
\section{Transaction Model}
\label{sec:transactions}
% ══════════════════════════════════════════════════════════════════════

\subsection{UTXO Accounting}

qBTC uses a UTXO model semantically identical to Bitcoin.  Each
transaction consumes unspent outputs and produces new ones; the input
surplus constitutes the miner fee.

\begin{definition}[Transaction]
$\mathrm{tx} = (\mathrm{inputs}, \mathrm{outputs}, \mathrm{body},
\mathrm{timestamp})$ where $\mathrm{body}$ contains the message
string \texttt{sender:receiver:amount:timestamp:chain\_id}, the
ML-DSA-87 signature~$\sigma$, and the public key~$pk$.  The
identifier is $txid = \mathrm{SHA256d}(\mathrm{serialize}(\mathrm{tx}))$
with JSON sorted-key serialisation.
\end{definition}

\subsection{Validation Rules}

\begin{enumerate}[label=(\roman*)]
  \item \textbf{Signature.}
    $\mathrm{ML\text{-}DSA.Verify}(pk, \mathrm{msg}, \sigma) =
    \texttt{true}$.
  \item \textbf{Key--address binding.}  Derived address from $pk$
    matches the sender in the message string.
  \item \textbf{Chain ID.}  Embedded chain ID matches the node's
    configured value---prevents cross-network replay.
  \item \textbf{Temporal freshness.}  Timestamp within 1~hour of
    current time (5-min future tolerance; relaxed during sync).
  \item \textbf{UTXO availability.}  All inputs exist and are
    unspent; concurrent mempool conflicts detected via pessimistic
    locking.
  \item \textbf{Value conservation.}
    $\sum\!\mathrm{in} \geq \sum\!\mathrm{out}$.
  \item \textbf{Amount bounds.}  Positive, $\leq$8 decimals,
    $\leq$21\,M.
\end{enumerate}


% ══════════════════════════════════════════════════════════════════════
\section{Economic Parameters}
\label{sec:economics}
% ══════════════════════════════════════════════════════════════════════

qBTC inherits Bitcoin's supply curve, adapted to a 10-second block
time:

\begin{table}[H]
\centering\small
\caption{Chain parameters}
\begin{tabular}{@{}lr@{}}
\toprule
\textbf{Parameter} & \textbf{Value} \\
\midrule
Total supply          & 21{,}000{,}000 \\
Mining allocation     & 10{,}500{,}000 (50\%) \\
Treasury allocation   & 10{,}500{,}000 (50\%) \\
Initial block reward  & 0.4167 qBTC \\
Halving interval      & 12.6\,M blocks \\
Block time target     & 10\,s \\
Diff.\ adjustment     & 120{,}960 blocks \\
Chain ID              & 1 \\
Min.\ relay fee       & 0.00001 qBTC \\
Max reorg depth       & 100 blocks \\
\bottomrule
\end{tabular}
\end{table}

The 50/50 mining--treasury split departs from Bitcoin's 100\% mining
allocation.  Bootstrapping a new proof-of-work chain requires initial
liquidity for development, audits, and ecosystem incentives.  The
treasury is fully transparent and subject to on-chain accounting from
the genesis block.

The halving interval of 12{,}600{,}000 blocks ($\approx$4~years at
10\,s) preserves Bitcoin's four-year halving rhythm.  The initial
reward is calibrated so that mining allocation is exhausted over 64
halvings, after which the network is sustained by transaction fees.


% ══════════════════════════════════════════════════════════════════════
\section{Network Architecture}
\label{sec:network}
% ══════════════════════════════════════════════════════════════════════

The qBTC peer-to-peer network operates across three layers: peer
discovery, data propagation, and mining coordination.

\subsection{Peer Discovery: Kademlia DHT}

Node discovery uses a Kademlia~[8] distributed hash table over UDP
(port~8001).  Each node announces its gossip endpoint, identity, and
NAT type.  New nodes bootstrap via a seed node and perform iterative
lookups to populate their routing table.  NAT traversal is supported
via STUN-based address detection and TCP hole-punching.

\subsection{Data Propagation: Gossip}

Blocks and transactions propagate via asynchronous gossip over TCP
(port~8002), using newline-delimited JSON (max 30\,MB).

\begin{table}[H]
\centering\small
\caption{Connection limits}
\begin{tabular}{@{}lr@{}}
\toprule
\textbf{Parameter} & \textbf{Value} \\
\midrule
Inbound connections    & 117 \\
Outbound connections   & 10 \\
Peers per /16 subnet   & 4 \\
Anchor peers           & 2 \\
\bottomrule
\end{tabular}
\end{table}

\textbf{Rate limiting} per peer: transactions 200/min, block
responses 30/min, control messages 120/min.  A broadcast tracker
deduplicates by hash.  \textbf{Reputation scoring} tracks protocol
compliance; peers below threshold are disconnected and banned.

\subsection{Mining Interface}

The RPC interface (port~8332) exposes \texttt{getblocktemplate},
\texttt{submitblock}, \texttt{getmininginfo}, and
\texttt{getblockchaininfo}, enabling compatibility with standard
mining software such as \texttt{cpuminer-opt}.


% ══════════════════════════════════════════════════════════════════════
\section{Security Model}
\label{sec:security}
% ══════════════════════════════════════════════════════════════════════

\subsection{Threat Taxonomy}

\begin{table}[H]
\centering\small
\caption{Attack classes and mitigations}
\begin{tabularx}{\columnwidth}{@{}lX@{}}
\toprule
\textbf{Attack} & \textbf{Mitigation} \\
\midrule
Sybil &
  Max 4 peers per /16 subnet; IP-based identity \\
Eclipse &
  Anchor peers; reputation scoring; DHT rediscovery \\
Deep reorg &
  100-block depth cap; cumulative-difficulty rule \\
DoS &
  Per-peer rate limits; Redis API limits; IP blocking \\
Double spend &
  Pessimistic UTXO locking; mempool conflict detection \\
Replay &
  Chain ID in signed message; signature domain separation \\
Timestamp &
  1-hour expiry; 5-min future tolerance \\
Key mismatch &
  Mandatory pubkey $\to$ address derivation check \\
\bottomrule
\end{tabularx}
\end{table}

\subsection{UTXO Integrity}

Double-spend prevention operates at two levels: the mempool
serialises UTXO access via pessimistic locking, and block
validation rejects intra-block conflicts.  No UTXO can be consumed
more than once, even under concurrent submission.

\subsection{Reorganisation Atomicity}

Reorgs execute as atomic \texttt{WriteBatch} operations with a
persistent completion marker.  On restart, presence or absence of the
marker deterministically indicates completion.  Incomplete reorgs are
rolled back automatically, preventing UTXO state corruption.


% ══════════════════════════════════════════════════════════════════════
\section{Cross-Chain Identity}
\label{sec:crosschain}
% ══════════════════════════════════════════════════════════════════════

Although qBTC is a sovereign chain with independent consensus, it
provides an optional mechanism for establishing cryptographic identity
linkage with Bitcoin.

\subsection{Commitment Protocol}

The commitment protocol binds a Bitcoin address to a qBTC address
through a Bitcoin-signed attestation:

\begin{enumerate}
  \item User constructs a message containing both addresses.
  \item User signs with their Bitcoin private key.
  \item Signed commitment is submitted to the qBTC network.
  \item After signature verification, the binding is recorded
    on-chain with bidirectional indexes.
\end{enumerate}

This establishes proof of identity without transferring value---a
verifiable record that a given Bitcoin key holder also controls a
specific qBTC address.  It serves as a foundation for future
interoperability protocols without introducing trust assumptions or
federation dependencies.

\subsection{Bitcoin Timestamping}

Commitments may optionally be timestamped to the Bitcoin blockchain
via OpenTimestamps~[10], producing a proof-of-existence anchored to a
Bitcoin block height.  This provides an independent temporal reference
resistant to qBTC chain reorganisations.

qBTC does not implement a direct value peg to Bitcoin.  Peg
mechanisms require either federated signers (trust assumptions) or
validity proofs (complexity and latency).  The commitment protocol
provides identity linkage cleanly, preserving the option of a bridge
as the technology matures.


% ══════════════════════════════════════════════════════════════════════
\section{State Management}
\label{sec:indexing}
% ══════════════════════════════════════════════════════════════════════

Storage uses RocksDB, a log-structured merge-tree store aligned with
blockchain access patterns (sequential writes, prefix scans).  An
optional Redis layer provides sub-millisecond access to hot data.

\subsection{Indexing}

Three specialised indexes are built incrementally:

\textbf{Block height index.}  Maps heights to block hashes; $O(1)$
retrieval via in-memory LRU cache backed by RocksDB.

\textbf{Wallet index.}  Per-address balance ($O(1)$ cached),
transaction history ($O(T)$), and UTXO set.  Updated atomically via
\texttt{WriteBatch}.

\textbf{Explorer index.}  The $k$ most recent non-coinbase
transactions ($k = 1{,}000$), sorted by timestamp, serving the
real-time explorer feed.

\subsection{Mempool}

Capacity: 10{,}000 transactions or 300\,MB.  Indexed by fee rate
and affected address.  When full, the lowest-fee-rate transaction is
evicted.  Per-address indexing enables $O(1)$ lookup of pending
transactions.


% ══════════════════════════════════════════════════════════════════════
\section{Client Architecture}
\label{sec:client}
% ══════════════════════════════════════════════════════════════════════

\subsection{Browser-Based Wallet}

qBTC ships a browser wallet in TypeScript/React.  Key generation and
signing use \texttt{@noble/post-quantum} (pure-JS ML-DSA-87).  Private
keys never leave the browser; they are AES-256-GCM encrypted before
storage.

\subsection{Real-Time Updates}

Persistent WebSocket connections deliver subscription-based updates:
\texttt{combined\_update} (wallet-scoped balance and history),
\texttt{all\_transactions} (global feed), and
\texttt{l1\_proofs\_testnet} (block proofs).  A 30-second keepalive
ping prevents proxy timeouts.

\subsection{Transaction Flow}

The wallet queries UTXOs, selects inputs covering amount plus 0.1\%
fee, constructs the message string
\texttt{sender:receiver:amount:timestamp:chain\_id}, signs with
ML-DSA-87, and submits.  The node validates, mempools, and gossips.


% ══════════════════════════════════════════════════════════════════════
\section{Deployment}
\label{sec:deployment}
% ══════════════════════════════════════════════════════════════════════

Two node configurations:

\textbf{Bootstrap server.}  Network hub: maintains the DHT, accepts
gossip, stores genesis, optionally mines.  Deployed behind Nginx with
TLS.

\textbf{Validator node.}  Full consensus participant: syncs history,
validates, gossips, optionally mines.  Independent of bootstrap.

Both expose ports 8080 (API), 8001 (DHT), 8002 (Gossip), 8332
(Mining RPC).  Docker Compose templates cover local testing (3-node),
bootstrap production, and validator production.  Prometheus metrics at
\texttt{/health}; Grafana dashboards for monitoring.

\textbf{Synchronisation.}  Nodes request missing blocks via gossip,
validate (PoW, signatures, UTXO state, Merkle root, difficulty),
add in height order with atomic index updates, and prune confirmed
transactions from mempool.  Timestamp checks are relaxed during sync.


% ══════════════════════════════════════════════════════════════════════
\section{Discussion: Why Not Wait?}
\label{sec:discussion}
% ══════════════════════════════════════════════════════════════════════

A natural objection is that qBTC is premature---that the quantum
threat is distant enough for an orderly migration within Bitcoin.
Three arguments suggest otherwise.

\textbf{Asymmetry of outcomes.}  If quantum computers arrive late,
early adoption costs nothing: post-quantum assets are no less secure
than classical ones.  If they arrive early, delayed migration is
catastrophic.  Prudent risk management favours early action when the
downside of inaction is unbounded.

\textbf{Governance bottleneck.}  Bitcoin's change process is, by
design, slow and conservative---a feature for a settlement layer, a
liability against cryptographic obsolescence.  A sovereign chain
sidesteps governance entirely: users opt in by choice, not by
committee.

\textbf{Harvest now, decrypt later.}  Nation-state adversaries are
collecting cryptographic material today for future decryption.  Public
keys exposed on Bitcoin's blockchain---through address reuse, P2PK
outputs, or signing---constitute a permanent, immutable record.  Every
day with ECDSA-signed UTXOs on a public ledger is a day the harvest
grows.

qBTC does not claim Bitcoin will fail.  It observes that the
migration path is uncertain, the timeline ambiguous, and the
consequences of miscalculation severe.  A sovereign quantum-safe
blockchain preserving Bitcoin's economic model is not a
competitor---it is a rational hedge against the most significant
cryptographic risk of this era.


% ══════════════════════════════════════════════════════════════════════
\section{Conclusion}
\label{sec:conclusion}
% ══════════════════════════════════════════════════════════════════════

qBTC demonstrates that post-quantum security and Bitcoin's economic
model are complementary.  By replacing ECDSA with ML-DSA-87 while
retaining SHA-256d proof-of-work, UTXO accounting, difficulty
adjustment, and the 21\,M supply cap, qBTC provides a sovereign
Layer-1 chain whose security does not erode with advances in quantum
hardware.

The system is operational: blocks are mined, transactions validated
and propagated, wallets sign with post-quantum keys, and the network
self-organises via DHT discovery and reputation-scored gossip.  The
architecture is conservative by choice: no novel consensus mechanism,
no unproven primitive, no speculative economic model.  The single
departure from Bitcoin is the one that matters.

The quantum threat to public-key cryptography is not a question of
\emph{if} but \emph{when}.  qBTC is built so that, when the answer
arrives, the question is already irrelevant.

\vspace{0.3cm}
\begin{center}
\small\textit{``The best time to migrate was yesterday. \\
The second best time is now.''}
\end{center}


% ══════════════════════════════════════════════════════════════════════
% References (spanning both columns)
% ══════════════════════════════════════════════════════════════════════
\balance   % balance final two columns

\vspace{0.4cm}
\noindent\rule{\columnwidth}{0.4pt}

{\small
\subsection*{References}

\begin{enumerate}[label={[\arabic*]},leftmargin=1.5em,nosep]
  \item P.\,W.\ Shor, ``Algorithms for quantum computation,''
    \textit{Proc.\ 35th FOCS}, IEEE, 1994.
  \item NIST, ``FIPS~204: ML-DSA,'' U.S.\ Dept.\ of Commerce, 2024.
  \item NIST, ``FIPS~205: SLH-DSA,'' U.S.\ Dept.\ of Commerce, 2024.
  \item S.\ Nakamoto, ``Bitcoin: A Peer-to-Peer Electronic Cash
    System,'' 2008.
  \item L.\,K.\ Grover, ``A fast quantum mechanical algorithm for
    database search,'' \textit{Proc.\ 28th STOC}, 1996.
  \item D.\,J.\ Bernstein and T.\ Lange, ``Post-quantum
    cryptography,'' \textit{Nature} 549, 2017.
  \item M.\ Mosca, ``Cybersecurity in an era with quantum
    computers,'' \textit{IEEE S\&P} 16(5), 2018.
  \item P.\ Maymounkov and D.\ Mazi\`{e}res, ``Kademlia,''
    \textit{IPTPS}, 2002.
  \item Open Quantum Safe, ``liboqs,'' 2024.
    \url{https://openquantumsafe.org}.
  \item P.\ Todd, ``OpenTimestamps,'' 2016.
    \url{https://opentimestamps.org}.
\end{enumerate}
}


% ══════════════════════════════════════════════════════════════════════
% Appendix (still two-column)
% ══════════════════════════════════════════════════════════════════════

\vspace{0.4cm}
\appendix

\section{Protocol Constants}
\label{app:constants}

\begin{table}[H]
\centering\small
\begin{tabular}{@{}lr@{}}
\toprule
\textbf{Constant} & \textbf{Value} \\
\midrule
\texttt{TOTAL\_SUPPLY}       & 21{,}000{,}000 \\
\texttt{MINING\_SUPPLY}      & 10{,}500{,}000 \\
\texttt{BLOCK\_REWARD}       & 0.4167 \\
\texttt{HALVING\_INTERVAL}   & 12{,}600{,}000 \\
\texttt{BLOCK\_TIME}         & 10\,s \\
\texttt{DIFF\_INTERVAL}      & 120{,}960 \\
\texttt{MAX\_REORG}          & 100 \\
\texttt{CHAIN\_ID}           & 1 \\
\texttt{MIN\_FEE}            & 0.00001 \\
\texttt{TX\_EXPIRY}          & 3{,}600\,s \\
\texttt{MAX\_MEMPOOL}        & 10{,}000 / 300\,MB \\
\texttt{MAX\_INBOUND}        & 117 \\
\texttt{MAX\_OUTBOUND}       & 10 \\
\texttt{SUBNET\_LIMIT}       & 4 per /16 \\
\texttt{ANCHOR\_PEERS}       & 2 \\
\bottomrule
\end{tabular}
\end{table}

\section{Key and Signature Sizes}
\label{app:keysizes}

\begin{table}[H]
\centering\small
\begin{tabular}{@{}lrr@{}}
\toprule
 & \textbf{ML-DSA-87} & \textbf{ECDSA} \\
\midrule
Public key  & 2{,}592\,B & 33\,B \\
Secret key  & 4{,}896\,B & 32\,B \\
Signature   & 4{,}627\,B & 72\,B \\
\bottomrule
\end{tabular}
\caption{The $\approx$64$\times$ signature increase is the
primary trade-off for quantum resistance.}
\end{table}

\end{document}
